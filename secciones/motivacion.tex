\section{Motivación}

El planteamiento del proyecto nace de un previo conocimiento de la \textit{Fundación Escuela de Solidaridad}. Durante muchos años, familia y conocidos han estado involucrados con el trabajo de la asociación, siendo conscientes de la labor que estos desempeñaban. Esto me ayudo a partir de un conocimiento previo de su situación, así como de obtener la motivación necesaria para afrontar el proyecto.

En su situación (explicada más ampliamente en la sección \ref{sec:asociacion}) es muy difícil gestionar todos sus residentes, personas asociadas y actividades mediante herramientas convencionales. Con cientas de personas y decenas de casas a su cargo, el no tener las herramientas adecuadas que les permitan clasificar, ordenar y filtrar provoca en ellos ralentizaciones o necesidad de un aumento del personal, situación que muchas veces no pueden asumir. 

El mercado que abarca este tipo de asociaciones no es lo bastante amplio, como para generar una rentabilidad que haga que empresas de tecnología inviertan en crear productos dirigidos a estas. Muchas veces es difícil encontrar herramientas tecnológicas rentables que puedan adaptarse a sus necesidades. En nuestro caso, la \textit{Fundación Escuela de Solidaridad} encuentra ciertos problemas en su flujo de trabajo al trabajar con herramientas menos especializadas como son hojas de cálculo u otros programas ofimáticos. 

Durante algunas conversaciones con ellos, estos remarcaban que soluciones del estilo realizadas a asociaciones similares de otros países, habían tenido un coste no asumible para esta.

Todo esto animó a diseñar y desarrollar una opción específica para su situación, que cubriese los problemas que buscaban solucionar, así como mejorar su flujo de trabajo de cara a facilitar su labor humanitaria.  