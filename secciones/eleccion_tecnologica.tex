\section{Elección tecnológica}

\subsection{Desarrollo móvil}

Uno de los requisitos que se contemplan en el proyecto especifica que la aplicación debe estar desarrollada tanto para Android como para IOS (\ref{rnf-plataformas}). Esto será una de las mayores condiciones a tener en cuenta a la hora de elegir las herramientas a utilizar para el desarrollo de la aplicación móvil. Actualmente para el desarrollo de aplicaciones móviles existen dos alternativas: las aplicaciones nativas y las aplicaciones denominadas como \textit{cross-platform}.

Las aplicaciones nativas, son las desarrolladas con las herramientas proporcionadas por las diferentes plataformas. Para Android tenemos Android SDK \cite{android-sdk}, la cual permite desarrollar aplicaciones tanto Android con Java, Klotin y C++. Para dispositivos de Apple tenemos IOS SDK \cite{ios-sdk}, que nos permiten desarrollar en Objective-C y Swift. 

Por otro lado, las aplicaciones \textit{cross-platform} son ``aplicaciones que bajo un mismo desarrollo pueden funcionar en múltiples plataformas'' \cite{cross-platform-comparacion}. Dentro de este grupo se incluyen tanto aplicaciones web cargadas desde un motor de renderizado web, como aplicaciones que bajo el mismo código son convertidas en nativas para ambas plataformas por igual. 

En los últimos años han aparecido varias herramientas \textit{cross-platform}, cambiando el paradigma de desarrollo de aplicaciones móviles. El principal beneficio de estas tecnologías es la reducción del tiempo de desarrollo, así como los costes. En este caso, el tiempo de desarrollo es vital. El tener un solo desarrollo en lugar de dos proyectos paralelos para la aplicación proporcionará una mayor fluidez al proyecto, así como una mayor unificación de la experiencia de usuario en ambas plataformas. A su vez escoger esta alternativa proporcionará un más fácil mantenimiento posterior de la propia aplicación. 

En el artículo \cite{cross-platform-comparacion} se estudian las diferencias de rendimiento que existen entre aplicaciones desarrolladas de forma nativa con las desarrolladas con populares frameworks \textit{cross-platforms}. El estudio nos muestra que según los usos, el uso de recursos puede ser un poco mayor para aplicaciones desarrolladas con estas herramientas. Aun con esto, la diferencia de rendimiento es asumible, ya que en este caso la aplicación no necesitará del uso de herramientas nativas del sistema que lacren el rendimiento. La mayoría de frameworks actuales nos permiten desarrollar aplicaciones que ofrecen una experiencia de usuario fluida y muy cercana a la de aplicaciones nativas.

Existen muchas herramientas actuales para desarrollar aplicaciones \textit{cross-platforms}. Según una encuesta realizada por JetBrains a diferentes desarrolladores \cite{jetbrains-survey}, el desarrollo de aplicaciones multiplataforma está liderado por dos alternativas, \textit{React-native} y \textit{Flutter}. Desarrolladas por Facebook y Google, ambas son utilizadas en infinidad de aplicaciones como son Airbnb o Discord en el primer caso o The New York Times o Ebay en el segundo. Ambas alternativas serían totalmente válidas para el desarrollo de esta aplicación.

Entre ambas la mejor elección en este caso será React-Native. Principalmente, el conocimiento previo y experiencia con JavaScript facilitará el proceso de desarrollo y reducirá el tiempo de formación en el uso de la herramienta. Además, durante alguna de las reuniones con los clientes, estos especificaron que en un futuro les gustaría que la aplicación se adaptase a una web para poder trabajar desde diferentes dispositivos. El que React-Native tenga su tecnología ``hermana'' para la web, \textit{React}, permitirá que en un futuro, la adaptación a una aplicación web, sea bastante más sencilla debido a la reutilización de la mayoría del código. 

\subsection{Base de datos}

Últimamente se han popularizado muchas alternativas a las base de datos relacionales. La mayoría de ellas se engloban en el grupo popularmente conocido como NoSQL. Son agrupadas de esta forma debido a que muchas de ellas comparten diferencias con las relacionales. 

Ambos tipos de bases de datos tienen ciertas ventajas e inconvenientes. En la mayoría de los casos estas diferencias vienen dadas por como están estrucutrados los datos y las propiedades que estas garantizan.

Una de las propiedades clave en nuestro caso será la validez de los datos introducidos. Es clave que al introducir datos estos se inserten sin errores y no dejen a esta en un estado inválido. Necesitamos que la alternativa escogida respete las propiedades ACID. Esta es una de las principales ``desventajas'' de las alternativas NoSQL.  ``Normalmente estas no soportan las transiciones ACID porporcionadas por las bases de datos relacionales'' \cite{NoSQLvsSQL_1}. Esto nos da una primera pista de que muchas de estas alternativas pueden no ser válidas en este proyecto.

Para tomar la decisión final será necesario analizar algunas de las alternativas NoSQL más utilizadas en estos ámbitos. Las bases de datos \textbf{key-value} son una de ellas. Básicamente siguen una estructura en la que los datos están asociados a una clave única. Estas permiten accesos e insecciones rápidas y ofrecen una mayor flexbilidad en los datos a almacenar. Por otra parte la estructura es muy simple y carece de relaciones, algo fundamental en nuestro caso, lo que nos lleva a descartar esta opción.

A partir de esta alternativa surgen las bases de datos \textbf{basadas en documentos}. Estas tienen la misma estructura que las \textit{key-value}, con la diferencia de que utilizan metadatos asociados a los documentos y permiten obtener estos, no solo en base a su clave única, si no también en base a su contenido. Estas proporcionan una mayor flexibilidad que las relacionales, permitiendo que los documentos no tengan que seguir una estrucutra fija como ocurre con las tablas en las relacionales. En este caso, no necesitamos esta flexibilidad ya que los datos a almacenar han sido claramente predefinidos con el cliente. Por otra parte, ``las bases de datos orientadas a documentos deben ser evitadas si la base de datos requiere de muchas relaciones'' \cite{NoSQLvsSQL_2}, lo que en nuesto caso hará que nos decantemos de nuevo por las relacionales.

Otras bases de datos NoSQL muy utilizadas, son las bases de datos basadas en grafos. Estas ponen el foco en las relaciones aunque internamente estén implementadas bajo alternativas anteriormente mencionadas. En nuestro caso, no tiene sentido modelar la información bajo un grafo ya que las relaciones, aun siendo importantes, no tienen el foco principal, sobretodo a la hora de acceder a la información.

Hay bastante más alternativas NoSQL, pero en su mayoría tienen como foco el BigData o la ingeniería de datos, por lo que no nos serán interesantes aquí. Una base de datos relacional nos proporcionará la robustez y funcionalidades necesarias en este caso, siendo la mejor alternativa a utilizar.

\subsection{Desarrollo backend}

Para el desarrollo backend, casi cualquier alternativa es válida para lo que se necesita. Existen alternativas casi infinitas para el desarrollo del servidor. Una de las más populares actualmente es Node.JS. Esta podrá proporcionar una estructura \textit{Full-JavaScript} al proyecto que provocará una mayor estandarización del código, además de que reducirá la necesidad de aprendizaje de cualquier otra tecnología, debido al conocimiento y uso previo de esta por mi. Además, esta tecnología será compatible con la mayoría de servicios serverless actuales, que podrán ser una alternativa tenida en cuenta a la hora de alojar el proyecto en base a reducir la necesidad de mantenimiento por parte de la organización. 