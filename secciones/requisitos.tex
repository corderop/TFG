%%%%%%%%%%%%%%%%%%%%%%%%%%%%%%%%%%%%%%%
% REQUISITOS DEL PROYECTO
%%%%%%%%%%%%%%%%%%%%%%%%%%%%%%%%%%%%%%%

\section{Requisitos}

\subsection{Requisitos funcionales}

\begin{enumerate}[start=1,label={RF-\arabic*.}]

    \item La aplicación contemplará tres secciones: gestión de personas, gestión de alojamiento y gestión de actividades.
    \item Cada usuario que use el sistema tendrá que identificarse con unas credenciales únicas.
    \item Existirán 4 roles de usuario que restringirán que actividades podrán o no hacer. Estos serán: administrador, estándar, invitado y participante.
    \item Los usuarios con el rol de administrador serán los únicos que podrán crear, modificar y eliminar a otros usuarios, así como modificar sus roles.
    \item Los usuarios con el rol de invitado podrán tener restringido el acceso a cualquiera de las secciones.
    \item Los usuarios con el rol de participante solo tendrán acceso a la sección de actividades.
    \item El sistema contemplará una opción de exportación de los datos de personas, de ocupación y de participación en actividades, a diferentes formatos (csv y excel).
    \item Las estadísticas (\ref{rf-es-personas}, \ref{rf-es-ocupacion}) podrán mostrarse de forma total o filtradas por periodos de tiempo (semanas, meses, años...).

\end{enumerate}

\subsubsection{Gestión de personas}

\begin{enumerate}[start=9,label={RF-\arabic*.}]

    \item Existirán cuatro tipos de personas en el sistema: residentes, voluntarios, socios y colaboradores.
    \item Se podrán añadir y eliminar personas del sistema.
    \item Se podrá modificar cualquier dato de las personas del sistema.
    \item Se podrán añadir y eliminar documentos asociados a una persona.
    \item Se podrán dar de alta y de baja a los residentes más de una vez. Se considera un alta cuando entran a residir en la fundación y una baja cuando la abandonan.
    \item La información de cada uno de los residentes se seguirá almacenando en el sistema aunque se le haya dado de baja de la fundación, con la excepción de que un administrador decida eliminar a esa persona del sistema. 
    \item Los usuarios con el rol de administrador podrán:
    \begin{itemize}
        \item Añadir y eliminar personas del sistema.
        \item Dar de alta y de baja a personas.
        \item Modificar los datos de una persona.
        \item Añadir y eliminar documentos asociados a una persona.
        \item Consultar los datos de cualquier persona del sistema.
        \item Crear y eliminar usuarios del sistema.
    \end{itemize}
    \item Los usuarios con el rol estándar podrán:
        \begin{itemize}
            \item Dar de alta y de baja a personas.
            \item Añadir personas al sistema.
            \item Consultar datos de las personas.
            \item Consultar los documentos de las personas.
        \end{itemize}
    \item Los usuarios con el rol invitado podrán:
        \begin{itemize}
            \item Consultar los datos de las personas
            \item Consultar los documentos de las personas
        \end{itemize}
    \item Los usuarios con el rol de participante no podrán acceder a esta sección.
    \item El acceso a la información por parte de los usuarios con el rol de invitados podrá estar limitada en cuanto a tipos de personas.
    \item El sistema contemplará búsquedas de personas por cualquiera de los datos asociados a estas o colectivos a los que pertenezcan.
    \item Se contemplarán búsquedas mixtas e individuales, es decir, búsquedas dentro de un tipo de persona concreto, o búsquedas de cualquier tipo de persona sobre campos coincidentes (búsqueda por nombre, ocupación etc.).
    \item Los documentos asociados a las personas serán de tipo imagen o documento de texto (pdf, doc, odt).
    \item A la hora de añadir una fotografía desde la aplicación, se proporcionará una cámara nativa además de incluir la opción de agregarla desde la galería de imágenes del smartphone.
    \item \label{rf-es-personas} La aplicación tendrá un apartado dedicado a mostrar estadísticas sobre:
        \begin{itemize}
            \item Número de residentes totales de la fundación.
            \item Número de residentes filtrados por género.
            \item Número de residentes menores.
            \item Número de residentes hijos de otro residente.
            \item Número de madres y padres.
            \item Número de residentes filtrados por colectivo al que pertenecen.
            \item Número de voluntarios dados de alta.
            \item Número de residentes dados de alta.
            \item Número de residentes dados de baja.
            \item Número de voluntarios.
            \item Número de socios nuevos.
            \item Número de personas empadronadas.
            \item Número de personas con documentación arreglada.
            \item Número de personas que se fueron por cada uno de los diferentes motivos.
        \end{itemize}
    \item Se contemplará un sistema de citas asociadas a cada residente de la fundación:
        \begin{itemize}
            \item Se almacenarán citas importantes asociadas a cada uno de los residentes.
            \item Se notificará a través de la aplicación a los usuarios con los roles de administrador y estándar un día antes de la cita y el mismo día de esta.
            \item Se notificará a por email al residente un día antes de la cita y el mismo día de esta.
        \end{itemize}

\end{enumerate}

\subsubsection{Gestión de alojamiento}

\begin{enumerate}[start=26,label={RF-\arabic*.}]

    \item Se podrán añadir, eliminar y modificar casas en el sistema.
    \item Se podrán añadir, eliminar y modificar habitaciones dentro de las casas.
    \item Se podrán añadir, eliminar y modificar residentes de las habitaciones.
    \item Se podrá cambiar de casa o habitación a los residentes.
    \item Existirá una ocupación para cada habitación.
    \item Los usuarios con el rol de administrador en tema de alojamientos podrán:
        \begin{enumerate}
            \item Añadir y eliminar casas.
            \item Añadir o eliminar habitaciones dentro de las casas.
            \item Modificar la casa o habitación de un residente.
            \item Consultar los residentes que ocupan una casa o habitación.
        \end{enumerate}
    \item Los usuarios con el rol estándar en tema de alojamientos podrán:
        \begin{enumerate}
            \item Asignar una casa o habitación a un residente que no tengan ninguna asignada ya.
            \item Consultar los residentes que ocupan una casa o habitación. 
        \end{enumerate}
    \item Los usuarios con el rol de invitados solo podrán ver quien ocupa cada casa/habitación.
    \item Los usuarios con el rol de participante no podrán acceder a la sección de alojamientos.
    \item \label{rf-es-ocupacion} Las estadísticas que debe cubrir la aplicación sobre los alojamientos serán:
        \begin{itemize}
            \item Porcentaje de ocupación actual de las casas.
            \item Número de plazas libres sobre el total.
            \item Número de casas con alguna plaza libre.
        \end{itemize}    

\end{enumerate}

\subsubsection{Gestión de actividades}

\begin{enumerate}[start=36,label={RF-\arabic*.}]

    \item Los usuarios con el rol de administrador podrán:
        \begin{itemize}
            \item Añadir nuevas actividades.
            \item Eliminar actividades.
            \item Ver actividades disponibles.
            \item Ver la información de cualquier actividad.
            \item Puntuar a los asistentes de una actividad.
        \end{itemize}
    \item Los usuarios con el rol estándar podrán:
        \begin{itemize}
            \item Añadir nuevas actividades.
            \item Ver actividades disponibles.
            \item Ver la información de cualquier actividad.
            \item Puntuar a los asistentes de una actividad.
        \end{itemize}
    \item Los usuarios con el rol de asistente podrán:
        \begin{itemize}
            \item Apuntarse a actividades.
            \item Desapuntarse de actividades.
            \item Ver actividades disponibles.
            \item Ver la información de cualquier actividad.
            \item Ver las estadísticas de su participación en actividades, así como el ranking con respecto a otros participantes.
        \end{itemize}
    \item Las estadísticas que cubrirá la aplicación para cada individuo será:
        \begin{itemize}
            \item Actividades realizadas
            \item Puntuación media de las actividades
            \item Puntuación individual en cada una de las actividades realizada.
        \end{itemize}
    \item Existirá un ranking de usuarios en base a la puntuación total de todas las actividades realizadas por cada uno de los usuarios.
    \item Existirán diferentes divisiones en base a la puntuación de cada usuario:
        \begin{itemize}
            \item Bronce: Menos de 10 puntos.
            \item Plata: Entre 10 y 50 puntos.
            \item Oro: Entre 50 y 150 puntos.
            \item Platino: Más de 150 puntos.
        \end{itemize}

\end{enumerate}

\subsection{Requisitos no funcionales}

\begin{enumerate}[start=1,label={RNF-\arabic*.}]
    \item La aplicación debe estar desarrollada tanto para Android como para IOS.
    \item Todos los usuarios, menos los del rol de invitado, estarán asociados a un residente.
    \item Todos los datos se transmitirán y almacenarán de forma segura, evitando así exponer información delicada.
    \item Los datos de carácter sensible no se almacenarán nunca en el dispositivo del usuario.
    \item Todo el sistema tendrá que cumplir con la Ley Orgánica 3/2018, de 5 de diciembre, de Protección de Datos Personales y garantía de los derechos digitales.
\end{enumerate}

\subsubsection{Gestión de personas}

\begin{enumerate}[start=6,label={RNF-\arabic*.}]

    \item Cada persona tendrá asociados datos personales, así como documentos.
    \item Los datos asociados de los residentes serán:
        \begin{itemize}
            \item Nombre y apellidos
            \item Documento identificativo (DNI, pasaporte...)
            \item Fecha de nacimiento
            \item Genero
            \item Email
            \item Teléfono móvil con prefijo
            \item Nacionalidades
            \item País de nacimiento
            \item País de procedencia
            \item Fotografía
            \item Fechas de ingreso
            \item Fechas de baja
            \item Motivo de baja
            \item Estado civil
            \item Colaborador allegado (Colaborador que ha traído a esta persona a la fundación)
            \item Formación (ESO, Bachillerato...)
            \item Ocupación
            \item Factores de riesgo
            \item Si tiene o no el permiso de trabajo
            \item Un valor identificativo del estado de su documentación española
            \item Un valor identificativo del estado de su empadronamiento en caso de que esté residiendo en ese momento
            \item Un valor identificativo del estado de su tarjeta sanitaria
            \item Residente de la fundación del que es hijo o hija
            \item Residente de la fundación del que es padre o madre
            \item Colectivos a los que pertenecen
        \end{itemize}
    \item Los datos asociados de los voluntarios serán:
        \begin{itemize}
            \item Nombre y apellidos
            \item Documento identificativo (DNI, pasaporte...)
            \item Fecha de nacimiento
            \item Género
            \item Email
            \item Nacionalidad
            \item Fotografía
            \item Dirección
            \item Ciudad
            \item País
            \item Nacionalidad
            \item Teléfono
            \item Conocimientos
            \item Preferencias
            \item Idiomas
            \item Disponibilidad de mañanas
            \item Disponibilidad de tardes
            \item Disponibilidad de fines de semana
            \item Fechas de llegada
            \item Fechas de salida
            \item Experiencias previas
        \end{itemize} 
        
    \item Los conocimientos de un voluntario/colaborador posibles serán: informática, diseño web, cocina, agricultura, guardería, albañilería, fontanería u otros en caso de que el usuario que los quiera especificar.
    \item Las preferencias de un voluntario serán seleccionadas sobre los conocimientos previamente indicados.
    \item Los datos asociados de los colaboradores serán:
        \begin{itemize}
            \item Nombre y apellidos
            \item Documento identificativo (DNI, pasaporte...)
            \item Fecha de nacimiento
            \item Género
            \item Email
            \item Nacionalidad
            \item Dirección
            \item Ciudad
            \item País
            \item Edad
            \item Teléfono
        \end{itemize}
    \item Los datos asociados a los socios serán:
        \begin{itemize}
            \item Nombre y apellidos
            \item Fecha de inicio
            \item Cuota
            \item Relación con otro socio
            \item Email
            \item Indicador si quiere recibir una newsletter
        \end{itemize}
    \item El sistema seguirá almacenando los datos de las personas a pesar de su baja en la fundación.
    \item Todos los teléfonos almacenados en el sistema tendrán el prefijo del país correspondiente.

\end{enumerate}

\subsubsection{Gestión de alojamiento}

\begin{enumerate}[start=15,label={RNF-\arabic*.}]

    \item Cada casa estará identificada por una letra única en base a las demás casas. En caso de tener más casas de las letras disponibles, se comenzarán a usar combinaciones de dos letras.
    \item Cada habitación estará identificada por un número único en base a las demás habitaciones dentro de la misma casa.
    \item Cada habitación tendrá un número de plazas y una ocupación (residentes actuales).

\end{enumerate}

\subsubsection{Gestión de actividades}

\begin{enumerate}[start=18,label={RNF-\arabic*.}]

    \item Cada actividad tendrá asociada la siguiente información:
        \begin{itemize}
            \item Nombre
            \item Foto de la actividad
            \item Descripción
            \item Fecha
            \item Asistentes de una actividad
            \item Puntuación de los asistentes a la actividad
        \end{itemize}
    \item La puntuación de un residente en una actividad será de un mínimo de 0 y un máximo de 10.

\end{enumerate}