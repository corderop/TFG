\thispagestyle{empty}

\begin{center}
{\large\bfseries Aplicación móvil para la Fundación Escuela de Solidaridad \\ Subtítulo }\\
\end{center}
\begin{center}
	Pablo Cordero Romero\\
\end{center}

%\vspace{0.7cm}

\vspace{0.5cm}
\noindent{\textbf{Palabras clave}: \textit{aplicación móvil}, \textit{cross-platform}, \textit{SCRUM}, \textit{Modelo Vista Controlador}, \textit{gestión de personas}, \textit{gestión de alojamientos}, \textit{acogida de personas}
\vspace{0.7cm}

\noindent{\textbf{Resumen}\\
	
%%%%%%%%%%%%%%%%%%%%%%%%%%%%%%%%%%%%%
% RESUMEN EN ESPAÑOL                %
%%%%%%%%%%%%%%%%%%%%%%%%%%%%%%%%%%%%%

La \textit{Fundación Escuela de Solidaridad} es una asociación sin ánimo de lucro que se dedica a la acogida y reinserción de personas en situación de exclusión social. Para ellos, gestionar y acceder a la información de todos sus beneficiarios y de personas relacionadas, como son socios, voluntarios y colaboradores, siempre ha sido un problema. La falta de herramientas específicas para su labor les llevó a proponer el desarrollo de este proyecto. 

La organización planteó la aplicación como una plataforma en la que se pudiese gestionar fácilmente la información de las personas relacionadas con esta, los alojamientos donde residen los beneficiarios y las actividades que estos realizan junto con un sistema de gamificación que fomentase la participación. 

Tanto por la forma de trabajo de ellos, como por la necesidad de un acceso rápido a la información, se decidió desarrollar una aplicación móvil para dar solución al problema planteado. Para hacer posible este desarrollo tanto en IOS como en Android bajo un mismo proyecto, se utilizó el framework \textit{cross-platform} React Native.  

Por otra parte, el sistema diseñado se sustenta bajo una arquitectura monolítica basada en el patrón Modelo Vista Controlador y bajo el uso de bases de datos relacionales. 

Basándose en la metodología ágil SCRUM, se realizó un proceso de desarrollo iterativo basado en el uso de diferentes roles e incluyendo reuniones periódicas entre los implicados.

%%%%%%%%%%%%%%%%%%%%%%%%%%%%%%%%%%%%%

\cleardoublepage

\begin{center}
	{\large\bfseries Mobile app for Fundación Escuela de Solidaridad}\\
\end{center}
\begin{center}
	Pablo Cordero Romero\\
\end{center}
\vspace{0.5cm}
\noindent{\textbf{Keywords}: \textit{mobile application}, \textit{cross-platform}, \textit{SCRUM}, \textit{Model View Controller}, \textit{people management}, \textit{house management}, \textit{foster care}
\vspace{0.7cm}

\noindent{\textbf{Abstract}\\

%%%%%%%%%%%%%%%%%%%%%%%%%%%%%%%%%%%%%
% RESUMEN EN INGLÉS                 %
%%%%%%%%%%%%%%%%%%%%%%%%%%%%%%%%%%%%%

The \textit{Fundación Escuela de Solidaridad} is a non-profit association committed to the refuge and reintegration of people in a situation of social isolation. For this association, organizing and accessing the information of all their beneficiaries and related people, such as members, volunteers and collaborators, has always been a problem. Due to the lack of specific tools for their work, they proposed the development of this project.
 
The organization conceived the application as a platform where the information about the people involved, the accommodation where the beneficiaries live and the activities they carry out could be easily managed with a gamification system to encourage participation.
 
Due to their working methods, as well as their need to have quick access to information, it was decided to develop a mobile application to provide a solution to this problem. The React Native cross-platform framework was chosen in order to make the development of this application possible in iOS and in Android, both under the same project. 
 
Furthermore, the designed system is based on a monolithic architecture based on the Model View Controller pattern and the use of relational databases.

Based on the SCRUM agile methodology, an iterative development process was carried out based on the use of different roles and including regular meetings between those involved.

%%%%%%%%%%%%%%%%%%%%%%%%%%%%%%%%%%%%%

\cleardoublepage

\thispagestyle{empty}

\noindent\rule[-1ex]{\textwidth}{2pt}\\[4.5ex]

D. \textbf{Nuria Medina Medina}, Profesor(a) del Departamento de Lenguajes y Sistemas Informáticos

\vspace{0.5cm}

\textbf{Informo:}

\vspace{0.5cm}

Que el presente trabajo, titulado \textit{\textbf{Aplicación móvil para la Fundación Escuela de Solidaridad}},
ha sido realizado bajo mi supervisión por \textbf{Pablo Cordero Romero}, y autorizo la defensa de dicho trabajo ante el tribunal
que corresponda.

\vspace{0.5cm}

Y para que conste, expiden y firman el presente informe en Granada a Julio de 2021.

\vspace{1cm}

\textbf{La directora: }

\vspace{5cm}

\noindent \textbf{Nuria Medina Medina}
