\chapter{Estado del arte}

\section{Estado del arte}

\textit{Fundación Escuela de Solidaridad} realiza su función desde hace más de 30 años. La forma de trabajar desde sus inicios hasta hoy en día ha ido evolucionando de forma paralela con el crecimiento de esta. Teniendo que trabajar con muchas más personas, ha dificultado en gran parte su labor. La llegada de herramientas informáticas ha facilitado en parte este trabajo, permitiendo que labores como la gestión de residentes se haga de forma más eficiente. 

Aun con esto, las asociaciones tienen que conformarse con herramientas genéricas utilizadas en muchos ámbitos, como son programas de ofimática u otras alternativas. Al no tener ánimo de lucro, estas asociaciones hacen que no sea rentable para desarrolladores o empresas desarrollar alternativas informáticas directamente dirigidas a este sector. 

En el caso de la \textit{Fundación Escuela de Solidaridad}, es difícil que existan alternativas o modelos de aplicación que contemplen exactamente lo que la fundación busca. Para cubrir las necesidades específicas solicitadas, habría que fijarse en sectores comerciales similares a lo demandado y partir de esas alternativas:

\begin{itemize}
    \item Para la gestión de alojamientos, podríamos buscar alternativas en la gestión de hoteles o gestión de clientes. Habría que adaptar el flujo de trabajo, considerando las habitación de hotel como habitaciones de la fundación y los clientes de este como los residentes.
    \item En cuanto a la gestión de actividades, existen varias alternativas que permiten la creación y organización de actividades. Lo difícil es que estas alternativas tengan una integración completa con las de gestión del alojamiento y personas.
\end{itemize}

Organizaciones que se mueven en este ámbito, normalmente sustentado económicamente por donaciones, tienen tres principales alternativas para la introducción y uso de aplicaciones en su flujo de trabajo: Software Libre, planes económicos dirigidos a este tipo de asociaciones, planes con bajo coste de aplicaciones actuales. 

La primera alternativa, aplicaciones \textbf{Software Libre} o con licencias abiertas podrían ser una alternativa muy atractiva para la fundación. Aunque no todas las licencias abiertas implican una distribución y uso libre del proyecto, normalmente esto es así. Esto permitiría reducir costes considerablemente, preocupándose solamente de costes de mantenimiento y alojamiento. Aquí una de las alternativas podría ser \textit{Qlo Hotel Commerce} \cite{qloapps}. Bajo una licencia MIT, su software puede ser usado y distribuido de forma libre. Aplicado a la fundación se podría usar de la forma anteriormente mencionada. En primer lugar podríamos enfocar a los clientes como residentes de la fundación y las habitaciones como las propias del hotel. 

\begin{figure}[htbp]
    \centerline{\includegraphics[scale=.5]{imagenes/estado_arte/qlo.png}}
    \caption{Interfaz gráfica de Qlo Hotel \& Booking Reservation}
    \label{fig}
\end{figure}

Esta alternativa podría funcionar aunque con ciertos límites. Entre sus limitaciones algunas como el no poder guardar información personalizada de las personas o que tengan que ser los ``clientes'' los que se asocien a una habitación podría limitar el trabajo de la fundación.

De aplicaciones para gestionar actividades hay más alternativas libres. La mayoría de alternativas ofrecen los que busca la asociación: gestión de actividades, cuentas para residentes, etc. Una de las características que no incluye ninguna de las alternativas libres analizadas es la puntuación y ranking de asistentes, uno de los aspectos diferenciales que la fundación busca perseguir. Algunas de las más interesantes son Agorakit, Gancio o Mobilizon.

En cuanto a alternativas no libres o de pago, es difícil encontrar alternativas asequibles o con planes de precio especializados para organizaciones sin ánimo de lucro. Esto se puede ver como algo normal debido a que está orientado principalmente a un sector comercial.

Finalmente, analizando varias alternativas y diferentes tipos de producto, creo que no existe uno que consiga satisfacer las necesidades de la fundación. Hay varios ``punto flacos'' que ningun software de los analizados cumple, los cuales formarán parte de las características diferenciales de la alternativa a realizar;

\begin{itemize}
    \item Adaptación total de los datos recogidos a cada una de las personas asociadas con la fundación.
    \item Conexión total entre las gestión de las personas de la fundación y su participación en las actividades propuestas por esta.
    \item Gestión de citas de personas, algo que no estaba contemplado en la mayoría de productos debido al estar representadas las personas más como clientes que como residentes.
    \item Búsqueda avanzada en base a las características de los usuarios.
    \item Acceso a estadísticas de la fundación en base a parámetros especificados por ellos mismos.
\end{itemize}

A partir de esto, habiendo analizando y escuchado las preferencias de la organización, se decidió realizar una aplicación móvil, buscando sobretodo la inmediatez de acceso a la información y la fácil gestión de esta.     

\section{Aplicaciones móviles}

Los ``smartphones'' o teléfonos inteligentes han cambiado el paradigma de acceso a internet. Desde 