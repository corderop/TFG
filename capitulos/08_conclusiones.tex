\chapter{Conclusiones y trabajos futuros}

En revisión de todos los objetivos planteados al inicio del proyecto, creo que se ha conseguido desarrollar una solución que los cumpla todos. El objetivo principal, en base a las exigencias del cliente ha sido cumplido. La gestión de personas, actividades y casas se ha cumplido con las exigencias impuestas.

En cuanto a los objetivos específicos fueron creados buscando el desarrollo de un producto correcto y de calidad. El primero de ellos creo que ha sido cumplido. Todos los datos existentes en el sistema y solicitados en las reuniones con los clientes han sido contemplados en la solución final. La herramienta permite la creación, acceso, actualización y eliminación de todos ellos.

El segundo de los objetivos hacía referencia a la comunicación entre la aplicación y el sistema de almacenamiento y gestión de los datos. En este caso también se ha cumplido mediante la implementación de sistemas de autenticación y seguridad de todas los puntos de acceso al servidor, permitiendo que ni personas externas ni usuarios sin los permisos correspondientes accedan a los diferentes recursos.

En tercer lugar, el desarrollo de la aplicación ha sido completo contemplando cada una de las secciones especificadas. En cuanto al cuarto objetivo, de igual forma, ha sido garantizado mediante el uso de los sistemas de seguridad implementados.

Creo que el trabajo ha sido correcto en base a lo especificado y acordado con los clientes. Las metodologías de desarrollo y la priorización de unas tareas sobre otras ha permitido que el producto final entregado cumpla los requisitos establecidos.

\section{Trabajos futuros}

Además de todo esto, el proyecto se ha orientado con la mirada de, en un futuro, poder ampliar el trabajo realizado durante este proceso. Las tecnologías utilizadas para el proyecto, permitirán tanto ampliar el proyecto a nuevas funcionalidades como adaptar fácilmente su interfaz a otros dispositivos de trabajo como pueden ser ordenadores.

La utilización de la tecnología como React Native, permitirá fácilmente su adaptación a aplicaciones web mediante su framework ``hermano'' React. Por otra parte, la API abre un abanico de posibilidades para el desarrollo e implementación de nuevas ideas para la \textit{Fundación Escuela de Solidaridad}.